\documentclass[12pt,a4paper]{article}

% Packages
\usepackage[utf8]{inputenc}
\usepackage[vietnamese]{babel}
\usepackage{amsmath}
\usepackage{amsfonts}
\usepackage{amssymb}
\usepackage{geometry}
\usepackage{titlesec}

% Page setup
\geometry{margin=1in}

% Title formatting
\titleformat{\section}{\Large\bfseries}{\thesection}{1em}{}
\titleformat{\subsection}{\large\bfseries}{\thesubsection}{1em}{}

\begin{document}

\title{Từ điển thuật ngữ Đại số Tuyến tính}
\author{}
\date{}
\maketitle

\section*{A}

\subsection*{Adjugate matrix - Ma trận phụ hợp}
Ma trận chuyển vị của ma trận các phần tử phụ đại số. Được sử dụng để tính ma trận nghịch đảo.

\textbf{Ví dụ:} Cho $A = \begin{bmatrix} 2 & 1 \\ 5 & 3 \end{bmatrix}$

Các phần tử phụ đại số (cofactor):
$$C = \begin{bmatrix} 3 & -5 \\ -1 & 2 \end{bmatrix}$$

Ma trận phụ hợp (chuyển vị của $C$):
$$\text{adj}(A) = C^T = \begin{bmatrix} 3 & -1 \\ -5 & 2 \end{bmatrix}$$

Công thức: $A^{-1} = \frac{1}{\det(A)} \cdot \text{adj}(A)$

\section*{D}

\subsection*{Determinant - Định thức}
Một giá trị vô hướng đặc trưng cho ma trận vuông, ký hiệu $\det(A)$ hoặc $|A|$.

\textbf{Công thức ma trận $2 \times 2$:}
$$\det\begin{bmatrix} a & b \\ c & d \end{bmatrix} = ad - bc$$

\textbf{Ý nghĩa:}
\begin{itemize}
    \item $\det(A) \neq 0$: Ma trận khả nghịch
    \item $\det(A) = 0$: Ma trận suy biến (không khả nghịch)
\end{itemize}

\textbf{Ví dụ đơn giản:}
$$A = \begin{bmatrix} 3 & 2 \\ 1 & 4 \end{bmatrix} \Rightarrow \det(A) = (3)(4) - (2)(1) = 12 - 2 = 10$$

Vì $\det(A) = 10 \neq 0$, ma trận $A$ khả nghịch.

\section*{I}

\subsection*{Identity matrix - Ma trận đơn vị}
Ký hiệu: $I$ hoặc $I_n$ (với $n$ là kích thước)

Ma trận vuông có các phần tử trên đường chéo chính bằng 1 và các phần tử khác bằng 0.

\textbf{Ví dụ:}
$$I_2 = \begin{bmatrix} 1 & 0 \\ 0 & 1 \end{bmatrix}, \quad I_3 = \begin{bmatrix} 1 & 0 & 0 \\ 0 & 1 & 0 \\ 0 & 0 & 1 \end{bmatrix}$$

\textbf{Tính chất:} $AI = IA = A$ với mọi ma trận $A$ tương thích.

\textbf{Ví dụ thực tế:}
$$\begin{bmatrix} 2 & 3 \\ 5 & 7 \end{bmatrix} \begin{bmatrix} 1 & 0 \\ 0 & 1 \end{bmatrix} = \begin{bmatrix} 2 & 3 \\ 5 & 7 \end{bmatrix}$$

Giống như nhân một số với 1: $a \times 1 = a$

\subsection*{Inverse matrix - Ma trận nghịch đảo}
\textbf{Tên tiếng Anh khác:} matrix inverse

Ký hiệu: $A^{-1}$

Đối với ma trận vuông $A$ khả nghịch, ma trận nghịch đảo $A^{-1}$ thỏa mãn:
$$AA^{-1} = A^{-1}A = I$$

\textbf{Công thức ma trận $2 \times 2$:}
$$A = \begin{bmatrix} a & b \\ c & d \end{bmatrix} \Rightarrow A^{-1} = \frac{1}{ad - bc}\begin{bmatrix} d & -b \\ -c & a \end{bmatrix}$$

\textbf{Ví dụ:}
$$A = \begin{bmatrix} 2 & 1 \\ 5 & 3 \end{bmatrix} \Rightarrow A^{-1} = \begin{bmatrix} 3 & -1 \\ -5 & 2 \end{bmatrix}$$

Kiểm chứng: $AA^{-1} = \begin{bmatrix} 1 & 0 \\ 0 & 1 \end{bmatrix}$

\textbf{Ý nghĩa thực tế:} Ma trận nghịch đảo giúp "hoàn tác" phép biến đổi. Giống như phép chia trong số học: nếu $2 \times 5 = 10$, thì $10 \div 5 = 2$. Tương tự, nếu $A\vec{x} = \vec{b}$, thì $\vec{x} = A^{-1}\vec{b}$.

\subsection*{Invertible matrix - Ma trận khả nghịch}
\textbf{Tên tiếng Anh khác:} nonsingular matrix, non-degenerate matrix

\textbf{Tên tiếng Việt khác:} ma trận không suy biến, ma trận đảo ngược được

Ma trận vuông có ma trận nghịch đảo.

\textbf{Điều kiện:} $\det(A) \neq 0$

\textbf{Tính chất:}
\begin{itemize}
    \item Tồn tại $A^{-1}$ sao cho $AA^{-1} = A^{-1}A = I$
    \item Các hàng/cột độc lập tuyến tính
    \item Rank đầy đủ (full rank)
\end{itemize}

\textbf{Ví dụ:}
$$C = \begin{bmatrix} 1 & 0 \\ 0 & 2 \end{bmatrix}$$

Kiểm tra: $\det(C) = (1)(2) - (0)(0) = 2 \neq 0$ $\Rightarrow$ khả nghịch!

Ma trận nghịch đảo: $C^{-1} = \begin{bmatrix} 1 & 0 \\ 0 & \frac{1}{2} \end{bmatrix}$

\section*{L}

\subsection*{Linear dependence - Phụ thuộc tuyến tính}
Các vector (hàng hoặc cột) được gọi là phụ thuộc tuyến tính nếu một vector có thể biểu diễn dưới dạng tổ hợp tuyến tính của các vector khác.

\textbf{Ví dụ đơn giản:}
$$\begin{bmatrix} 1 & 2 \\ 2 & 4 \end{bmatrix}$$

Hàng 2 = $2 \times$ Hàng 1: $(2, 4) = 2 \times (1, 2)$ $\Rightarrow$ phụ thuộc tuyến tính.

\textbf{So sánh thực tế:} Giống như có 2 phương trình giống nhau:
\begin{align*}
x + 2y &= 5 \\
2x + 4y &= 10 \quad \text{(chỉ là nhân đôi phương trình trên)}
\end{align*}

Không có thông tin mới!

\subsection*{Linear independence - Độc lập tuyến tính}
Các vector (hàng hoặc cột) được gọi là độc lập tuyến tính nếu không vector nào có thể biểu diễn dưới dạng tổ hợp tuyến tính của các vector còn lại.

\textbf{Điều kiện:} Ma trận có các hàng/cột độc lập tuyến tính $\Leftrightarrow$ $\det(A) \neq 0$

\textbf{Ví dụ đơn giản:}
$$\begin{bmatrix} 1 & 0 \\ 0 & 1 \end{bmatrix}$$

Không hàng nào là bội số của hàng kia: $(1, 0)$ và $(0, 1)$ hoàn toàn khác nhau $\Rightarrow$ độc lập tuyến tính.

\textbf{So sánh thực tế:} Giống như có 2 phương trình khác nhau:
\begin{align*}
x &= 3 \\
y &= 5 \quad \text{(mỗi phương trình cho thông tin mới)}
\end{align*}

\section*{N}

\subsection*{Nonsingular matrix - Ma trận không suy biến}
Xem: \textit{Invertible matrix - Ma trận khả nghịch}

\section*{R}

\subsection*{Rank - Hạng}
Số lượng hàng (hoặc cột) độc lập tuyến tính tối đa của ma trận.

\textbf{Ký hiệu:} $\text{rank}(A)$

\textbf{Tính chất:}
\begin{itemize}
    \item Ma trận $n \times n$ khả nghịch $\Leftrightarrow$ $\text{rank}(A) = n$ (rank đầy đủ)
    \item Ma trận suy biến $\Leftrightarrow$ $\text{rank}(A) < n$ (rank khuyết)
\end{itemize}

\textbf{Ví dụ đơn giản:}
$$D = \begin{bmatrix} 1 & 2 & 3 \\ 0 & 4 & 5 \\ 0 & 0 & 6 \end{bmatrix}$$

Có 3 hàng độc lập tuyến tính $\Rightarrow$ $\text{rank}(D) = 3$ (rank đầy đủ).

$$E = \begin{bmatrix} 1 & 2 \\ 2 & 4 \end{bmatrix}$$

Chỉ có 1 hàng độc lập $\Rightarrow$ $\text{rank}(E) = 1 < 2$ (rank khuyết).

\section*{S}

\subsection*{Singular matrix - Ma trận suy biến}
\textbf{Tên tiếng Anh khác:} degenerate matrix, non-invertible matrix

\textbf{Tên tiếng Việt khác:} ma trận không khả nghịch, ma trận không đảo ngược được

Ma trận vuông không có ma trận nghịch đảo.

\textbf{Điều kiện:} $\det(A) = 0$

\textbf{Đặc điểm:}
\begin{itemize}
    \item Không tồn tại $A^{-1}$
    \item Các hàng/cột phụ thuộc tuyến tính
    \item Rank khuyết (rank-deficient): $\text{rank}(A) < n$
\end{itemize}

\textbf{Ví dụ:}
$$B = \begin{bmatrix} 1 & 2 \\ 2 & 4 \end{bmatrix}$$

Hàng thứ hai = $2 \times$ hàng thứ nhất $\Rightarrow$ $\det(B) = 0$ $\Rightarrow$ suy biến.

\textbf{Hệ quả thực tế:} Nếu giải hệ phương trình $B\vec{x} = \vec{b}$:
\begin{itemize}
    \item Hoặc vô nghiệm (không có lời giải)
    \item Hoặc vô số nghiệm (quá nhiều lời giải)
    \item Không bao giờ có duy nhất 1 nghiệm!
\end{itemize}

\subsection*{Square matrix - Ma trận vuông}
Ma trận có số hàng bằng số cột ($n \times n$).

Chỉ có ma trận vuông mới có định thức và ma trận nghịch đảo.

\textbf{Ví dụ:}

Ma trận vuông:
$$\begin{bmatrix} 1 & 2 \\ 3 & 4 \end{bmatrix}_{2 \times 2}, \quad \begin{bmatrix} 1 & 2 & 3 \\ 4 & 5 & 6 \\ 7 & 8 & 9 \end{bmatrix}_{3 \times 3}$$

KHÔNG phải ma trận vuông:
$$\begin{bmatrix} 1 & 2 & 3 \\ 4 & 5 & 6 \end{bmatrix}_{2 \times 3}, \quad \begin{bmatrix} 1 & 2 \\ 3 & 4 \\ 5 & 6 \end{bmatrix}_{3 \times 2}$$

\end{document}
